\documentclass[12pt,a4]{article}
\usepackage[utf8]{inputenc}
\usepackage{natbib}
\usepackage[yyyymmdd,hhmmss]{datetime}
\usepackage{graphicx}

\title{Review of literature on non-stationary tidal components with application to tide-gauge observations in the port of Esbjerg, Denmark.}
\author{Peter Thejll}
\date{{\today}  at {\currenttime}}

\begin{document}

\maketitle

\section{Introduction}
This is an attempt at an exhaustive collection of literature on the subject of non-stationarities in tidal components - in particular M2 in the North Sea, and especially for the port of Esbjerg in Denmark.


\section{1956}
The propagation of storm surges along the North Sea coastlines are discussed by~\cite{Darbyshire1956}.
The analysis is constrained to looking at the differences between observed and predicted tides, and the influence of meteorological parameters (i.e. sea-level air pressure) is discussed. There is no discussion of long-term changes in tidal constituents. Esbjerg data from the 1950s are used.

\section{1988}
\cite{Pugh1988} is cited~\citep{Dimon1997a} as noting non-statioanrity in tides, but whether just in levelor also standard deviation is unknown.

\section{1997}
In an analysis of the power-law nature of spectra of temporally high-resolution sea-level variations (1889-1994), \cite{Dimon1997a} note that Esbjerg hourly sea-level values display a growing annual mean  and an increasing standard deviation.

\section{1999}
\cite{Bijl1999} note that the homogeneity of daily-max data for Esbjerg has not been established, and that the effects due to deepening of the sailing channel from the North Sea to the port of Esbjerg is somewhat unknown. They note, however, that DHI has studied the problem and that they concluded no serious effect of imposed changes in depth, from 4.5 meters to 10 since 1850, on high-water levels during westerly storms. The trend in mean sea level at Esbjerg is also noted here. Also noted and discussed is the absence of maxima above 3m in these data before 1910. There is no discussion of non-stationarity in tidal components.

\section{2001}
\cite{Huess2001grl} study the seasonal variations in tidal components at Esbjerg, due essentially to gravitational effects acting on the anomalistic and tropical annual periods. Altimetric data from the TOPEX/Poseidon space platform are used, and the analysis is therefore restricted to a few years. Other non-stationarities in the tidal components such as those due to meteorological influences are discussed, and modelled.


\section{2002}
Future expected changes in waves and storm surges are studied by~\citep{Debernard2002} using projections of projected meteorological conditions in a future period. Tidal components are not studied as such; they are in fact subtracted.

\section{2005}
Increase in mean sea level is the main focus of~\cite{go05100g}. There is no discussion of non-stationarity of tidal components.

\section{2006}
\cite{Lyard2006} is a model study of the effect of bottom friction, etc,. on tidal constituents. It is mainly about the model software itself, and no comments on non-stationarity in M2 amplitudes are made, nor is there any mention of Esbjerg. Bottom friction is presented globally with a map showing some detail in the North Sea.

\section{2007}
\cite{Parker2007} is a detailed but general lecture book. No specific mention of Esbjerg is found.

\section{2009}
\cite{Madsen2009} studies sea levels around Danish coasts, including Esbjerg, but there is no discussion of tidal components.

\section{2010}
Extreme value distributions for sea levels in the North Sea (Cuxhaven) are studied by~\cite{Mudersbach2010}. Stationarity of data series is studied. GEV modelling is employed allowing for time-dependent parameters. Non-stationarity in sea-level extremes is detected. No separate study of tidal range non-stationarity is discussed.

\cite{Wahl2010a} study non-stationarity in sea level and tidal range (the k-factor) for Cuxhaven and Heligoland. The paper comments mainly on SLR, but stationarity of the k-factor is discussed and tested - it is apparently stationary for Heligoland and Cuxhaven. The k-factor is based on monthly data. No mention of Esbjerg.

\cite{Menendez2010} apply a nonstationary extreme value analysis  to the maxima of the total elevations and surges for the period of1970 and onward. Esbjerg data from GESLA is used. A simple method is applied to conclude "that much of the change in the extremes is due to change in the mean values."


\section{2012}
\cite{Pickering2012} model the effect of future SLR on tidal constituents such as the M2. The M2 tidal response is non-linear between 2 and 10 m with respect to SLR, particularly in the North Sea.
Under the 2 m SLR scenario the M2 constituent is particularly responsive in the resonant areas of  the southeastern German Bight and Dutch Wadden Sea (with large amplitude increases). With SLR the depth, wave speed and wave length (tidal resonance characteristics) are increased causing changes in near resonant areas. In   shallow areas SLR causes reduced energy dissipation by bottom friction. A map of the North Sea is shown indicating expected rises in M2 amplitude for Esbjerg with 2m SLR. Ribe and Hanstholm discussed in tables. Not Esbjerg. Effect on M2 phase also mentioned.

\cite{Idier2012} find that data analysis on the scale of several decades indicates that the largest instantaneous storm surges occur preferentially at low and rising tide at Dunkerque, with regional or local differences. They investigate the nature of the tide storm-surge dependence with respect to 'skew surges', and find that there are locations in the English Channel where skew surge is dependent on tide. No discussion of Esbjerg, nor of non-stationarity in tidal constituents.

\section{2013}
\cite{Mudersbach2013} \textbf{Needs to be fetched and read.}

\section{2014}
\cite{Dangendorf2014} study the homogeneity of a storm surge index from 1843-present, and find that it is stationary. Comments on higher uncertainties in the earlier years of the 20CRv2 over the North Sea region are made. No mention of Esbjerg or non-stationary tidal component evolution.

\cite{Grawe2014} look at the temporal evolution of the amplitudes of the M2 and M4 constituents in data (since 1920) from The North Sea as close as the Wadden Sea German islands. Phase and amplitude changes are compared for open-sea stations and more sheltered stations. Largest changes for the sheltered station data. Seasonal cycles in amplitude are looked at.  Due to the interaction of M2 and M4 amplitudes and phases, a seasonal signal in the residual sediment transport is induced. 

\section{2015}
\cite{Arns2015} is about non-stationary extreme-value modelling (POT) of storm surges in the German Bight, aided by model studies of tides. The response of the tidal propagation to SLR is investigated based on the results from a tidal analysis of individual storm surge events. They find an increase in the M2 amplitude and decrease in the amplitudes of frictional and overtides accompanied by less tidal wave energy dissipation. Changes in high water occurrence times due to SLR are discussed with a  simple model. SLR should increase the tidal wave speed as the tidal wave speed only depends on the
water depth for small amplitude waves in shallow water (i.e. no friction effects). M2, M3 and K1 do not appear to respond, in the model study, in the same way to SLR and the reason is discussed.

\cite{Marcos2015}: Decadal to multidecadal variations in sea level extremes unrelated to mean sea level changes are investigated. Esbjerg data are used since 1950. Key findings are 'intensity and the frequency of occurrence of extreme sea levels unrelated to mean sea level vary coherently on decadal scales in most of the sites' and 'extreme sea level changes are regionally consistent, thus pointing toward a common large-scale forcing.' Non-stationarity in M2 is not discussed. 


\section{2017}
Risk mapping of future extremes in sea-level are discussed by~\cite{Thorarinsdottir2017} for Bergen and Esbjerg. In this paper a statistical method for propagating uncertainties is described.

\cite{Devlin2017c} suggest that "in many regions, local flood level determinations should consider the joint effects of non-stationary tides and mean sea level (MSL) at multiple time scales."

\cite{Kuang2017} \textbf{need to acquire and read.}

\cite{Idier2017} model the effect of SLR on tidal range in the North Sea. Notable increases in high tide levels occur especially in the northern Irish Sea, the southern part of the North Sea and the German Bight. Changes are proportional to SLR, as long as SLR remains smaller than 2 m. Effect of allowing 'flooding' in the modelling approach is studied. Flooding reduces the tidal range change due to SLR. A preliminary estimate of tidal changes by 2100 under a plausible non- uniform SLR scenario (using the RCP4.5 scenario) is provided. Maps including Esbjerg are shown.

\cite{Lee2017} model the impact of SLR on US East Coast Estuaries and bays, especially in terms of what sea-walls which deny flooding do to the regional tidal range. \cite{Ross17} study the same  and have similar results.

\section{Summary}

The first to find evidence, as far as I can tell, that the tide at Esbjerg is non-stationary were~\cite{Dimon1997a}. Application of non-stationary statistical methods to extreme sea-levels for the North Sea or Esbjerg is recent~\citep{Mudersbach2010} although there is general appreciation of the need in textbook form earlier~\citep{Coles2001}. The understanding that tidal range could depend on depth of the water is present in general terms in early literature, but  specific model-studies of the effect of projected SLR on tidal range is not made until recently~\cite{Pickering2012,Arns2015,Devlin2017c,Idier2017,Lee2017}.

For Esbjerg, these latter studies all agree that tidal range of M2 in Esbjerg should increase with SL.

 

\bibliographystyle{alpha}
\bibliography{references}
\end{document}
